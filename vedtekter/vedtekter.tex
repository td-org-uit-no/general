\documentclass[11pt]{article}

\usepackage[utf8]{inputenc}
\usepackage{enumitem}
\usepackage{fullpage}

\setlist[enumerate]{label=\textbf{\alph*})}

\renewcommand{\thesection}{\S\arabic{section}}

\title{\Huge{Vedtekter for\\Tromsøstudentenes Dataforening}}
\date{7.\ september 2018}
\author{}

\begin{document}
\maketitle
\section{Generelle bestemmelser}
\subsection{Navn}
Foreningens navn er Tromsøstudentenes Dataforening, heretter forkortet TD.

\subsection{Formål}
TDs fremste oppgave er å fremme sosialt samhold blant studentene tilhørende institutt for informatikk.
TD skal også tilstrebe å tilby et bredt spekter av både faglige og sosiale arrangementer. 

\subsection{Taushetsplikt}
Medlemmene av TD har taushetsplikt i alle saker og henvendelser der personvern er nødvendig. 

\section{Årsmøtet}
\subsection{Formål}
Årsmøtet er det øverste organet for medlemmene i TD.

\subsection{Generelle bestemmelser}
\begin{enumerate}
	\item Årsmøtet skal avholdes i vårsemesteret.
	\item Innkalling til årsmøte skal offentliggjøres senest to uker før møtet. Sakspapirer offentliggjøres én uke før møtet. 
	\item Alle studenter som er registrert medlem i TD møter med tale-, forslags- og stemmerett. 
	\item Årsmøtet er vedtaksdyktig dersom det er minimum 10 studenter som beskrevet i §2-2c.
\end{enumerate}

\subsection{Saker som behandles på årsmøtet}
Årsmøtet skal behandle:
\begin{enumerate}
	\item Godkjenning av innkalling
	\item Godkjenning av dagsorden
	\item Valg av ordstyrer, referent og to protokollunderskrivere
	\item Sakene på dagsorden
	\item Valg av nytt styre til TD jf. §4-1 og §4-2. 
\end{enumerate}
Studenter kan sende inn saker til årsmøtet inntil åtte dager før møtet avholdes. Forslag til vedtak som kommer under møtet må komme skriftlig til møteleder.

\subsection{Ekstraordinært årsmøte}
Ekstraordinært årsmøte skal avholdes ved skriftlig krav fra minst 5 medlemmer i TD.\\
Ekstraordinært årsmøte skal avholdes på samme måte som et ordinært årsmøte. Bare saker fremmet før fristen kan behandles. 

\section{Valg og supplering}
\subsection{Valgbare studenter}
Det er kun studenter som er stemmeberettigede ved årsmøtet som kan stille til valg, jf. §2-2c.

\subsection{Gjennomføring av valg}
\begin{enumerate}
	\item Kandidater velges ved simpelt flertall
	\item Dersom minst én i forsamlingen krever det skal det avholdes skriftlig valg
	\item Valg gjennomføres i rekkefølge jamfør opplisting i Kapittel 4
	\item En person kan ikke velges til mer enn ett verv i TDs styre
	\item Alle verv velges for én årsmøteperiode
	\item Kandidater skal gis anledning til en kort presentasjon
\end{enumerate}

\subsection{Årsmøteperiode}
En årsmøteperiode defineres som perioden fra og med 1.\ august inneværende år til og med 31.\ juli påfølgende år.

\subsection{Diversifisering}
Det skal tilstrebes en difersifisering av medlemmene i TD, være seg kjønn og alder/kull.

\subsection{Supplering av TD}
Hvis det oppstår en situasjon der styret ikke er fylt opp på årsmøtet, eller noen fratrer sitt verv i løpet av årsmøteperioden kan det resterende styret selv avgjøre om de ønsker å supplere seg selv eller arrangere et ekstraordinært årsmøte.

\section{Organisasjon}
\subsection{Styret}
Ledelsen i TD heter styret og består av
\begin{enumerate}
	\item én leder
	\item én nestleder
	\item én økonomiansvarlig
	\item én eller to arrangementsansvarlige
	\item én teknisk ansvarlig
	\item én eller to kommunikasjonsansvarlige 
	\item én nettsideansvarlig
\end{enumerate}

Styret og medlemmene vil avgjøre selv hvorvidt det skal være en eller to arrangements-, og kommunikasjonsansvarlige. Dette gjøres ved valg av styret under årsmøtet. Alle styremedlemmer har taushetsplikt om de opplysninger de mottar som styremedlemmer.

\subsection{Arbeidsbeskrivelse}
For alle styremedlemmer forventes det at man møter opp på møter og aktiviteter i regi av TD, samt utfører de arbeidsoppgaver man har påtatt seg. I tillegg gjelder spesifikke arbeidsoppgaver:

\subsubsection{Leder}
Leder fungerer som leder for både styret og TD. Leder fungerer som møteleder på alle møter. Leder har stemmerett i styret og TD. Leder har øverste ansvar for økonomien og aktivitet i regi av TD.

\subsubsection{Nestleder}
Nestleder fungerer som leders stedfortreder. Dersom leder ikke har mulighet til å utføre sine plikter er det nestleders ansvar å utføre disse. Nestleder har stemmerett i styret og TD. Nestleder fungerer vanligvis som referent.

\subsubsection{Økonomisk ansvarlig}
Økonomisk ansvarlig har hovedansvar for TDs økonomi. Dette gjelder både føring av budsjett, regnskap og søknader om økonomisk støtte; gjerne med hjelp fra andre styremedlemmer.

\subsubsection{Arrangement ansvarlig}
Arrangementansvarlig har hovedansvaret for planlegging og gjennomføring av arrangementer i regi av TD. 
Arrangementansvarlig(e) kan også sette ned en arrangementskomité for hvert spesifikke arrangement.

I henhold til §4.1 vil det være mulig å innsette to arrangementansvarlige.

\subsubsection{Teknisk ansvarlig}
Teknisk ansvarlig har hovedansvar for teknisk drift i regi av TD. Dette inkluderer drift av serverrom, andre tekniske installasjoner, samt teknisk bistand ved arrangmenter.

\subsubsection{Kommunikasjons ansvarlig}
Kommunikasjonsansvarlig(e) har hovedansvar for TDs kommunikasjon utad. 
Denne kommunikasjonen inkluderer tilgjengeliggjøring og vedlikehold av informasjonen som finnes om foreningen og dens virke.

I henhold til §4.1 vil det være mulig å innsette to kommunikasjonsansvarlige. Ansvaret vil da deles i to områder; bedriftskommunikasjon og studentkommunikasjon.

\subsubsection{Nettsideansvarlig}
Nettsideansvarlig har hovedansvar for å fremme drift og utvikling av TD sin nettside. Nettsideansvarlig har overordnet ansvar for å legge ut produsert innhold på nettsiden, samt bistå ved komplikasjoner med drift av nettsiden.

\subsection{Møtevirksomhet}
\subsubsection{Intern møtevirksomhet}
\begin{enumerate}
	\item TD avholder møter etter behov, minst tre ganger pr. semester.
	\item Leder eller minst tre medlemmer kan kreve innkalt til TD-møte.
	\item Møtetidspunkt skal gjøres kjent senest én uke før møtet avholdes.
	\item Forfall skal meldes til leder eller møteinnkaller.
	\item TD er vedtaksdyktige når minst halvparten av styremedlemmene er til stede.
	\item Nestleder eller annen referent har ansvar for å føre referat og at det sendes ut til medlemmene senest to dager etter møtet. Det er en høringsfrist på to virkedager. Etter fristen skal referatet offentliggjøres. Dermed kan referatet ikke inneholde informasjon unntatt fra offentligheten.
	\item Møtene er åpne, dersom man ikke vedtar å lukke ett møte.
	\item Simpelt flertall er tilstrekkelig for å gjøre vedtak i en sak. Ved stemmelikhet har leder dobbeltstemme.
	\item Avstemning skjer ved håndsopprekning, dersom ingen krever skriftlig avstemning.
	\item Dersom et medlem er sterkt uenig i et vedtak, kan vedkommende kreve å få det protokollført.
	\item Styret kan gis fullmakt til å gjøre vedtak i hastesaker.
\end{enumerate}

\subsubsection{Årsmøtet}
TD skal avholde et årsmøte én gang tidlig på høstsemesteret. Hensikten med årsmøtet er å velge et nytt styre, samt å opprettholde kontakt med medlemsmassen. Etter endt årsmøte skal det sendes inn styreendringer til Brønnøysundregistrene i henhold til deres instrukser.

\subsection{Bedriftaktiviteter}
TD skal ta et honorar for alle typer aktiviteter med bedrifter der TD benytter sin posisjon for å sette bedriften i kontakt med TDs studenter. Ved spesielle tilfeller der styret ved kvalifisert flertall (2/3-flertall) er enige om et unntak kan dette gjøres, noe som må reflekteres i møtereferatet. TD-styret fastsetter honorarsatser. 

\subsection{Æresmedlem}
Æresmedlem er den høyeste utmerkelsen i foreningen. Et æresmedlem gis utover et vanlig medlemskap av foreningen heder og ære for sitt virke for foreningen.

\subsubsection{Utnevnelse av æresmedlem}
Æresmedlem skal nomineres i forkant av et årsmøte, hvor en vurdering vil bli gjort av styremedlemmene i TD. Æresmedlem er ikke begrenset til medlemmer eller tidligere medlemmer av foreningen, men utnevnes etter vurdering av følgende kriterier:

\begin{enumerate}
	\item En sterk ressurs som har utøvd betydelig arbeid utover forventningene av sin rolle i foreningen.
	\item Har gjennom sitt arbeid lagt til rette særlig fortjeneste for foreningen.
\end{enumerate}

Det presiseres viktighet i vurderingen av kriteriene, slik at inflasjon av æresmedlemskap unngås.

\subsection{Medlemskap}
Medlemskap av Tromsøstudentenes Dataforening er forbeholdt studenter ved studieretninger tilknyttet Institutt for Informatikk, UiT. Dette gjelder også tidligere studenter ved ovennevnte studieretninger. 

\subsubsection{Kontingent}
Det betales ikke medlemskontingent for medlemmer av Tromsøstudentenes Dataforening.

\subsubsection{Oppdatering av medlemsstatus}
Tromsøstudentenes Dataforening må årlig oppdatere sin medlemsinformasjon.
Det forventes at medlemmer i Tromsøstudentenes Dataforening oppdaterer sin status som enten uteksaminert, kommende eller aktiv student, samt årskull og annen informasjon som angår Tromsøstudentenes Dataforening.

\section{Sanksjoner}
\subsection{Iverksettelse av sanksjoner}
\begin{enumerate}
	\item Sanksjoner skal iverksettes mot de som fremstår som vesentlig uegnet til vervet de innehar.
	\item Før sanksjoner blir vedtatt skal den det gjelder få en advarsel om gjeldende forhold. Dette kan fravikes i særs grove tilfeller.
	\item Før sanksjoner blir iverksatt skal den det gjelder få anledning til å uttale seg for TD.
	\item Den det sanksjoneres mot har krav på skriftlig begrunnelse for vedtaket.
\end{enumerate}

\subsection{Mot medlemmer av TD}
\begin{enumerate}
	\item Dersom det er simpelt flertall i styret for å vedta at § 5-1a får anvendelse mot et styremedlem, skal vedkommende anmodes om å fratre. Leder skal foreta denne anmodningen, selv om vedkommende stemte mot vedtaket.
	\item Dersom styremedlemmet velger å fratre kan styret supplere seg selv ved å konstituere et nytt medlem frem til neste ordinære årsmøte, jf. § 3-4.
	\item Dersom medlemmet ikke ønsker å fratre skal det innkalles til ekstraordinært årsmøte, og reises mistillitsforslag der.
\end{enumerate}

\subsection{Mot leder av TD}
Dersom det foreligger 2/3-flertall i styret for å vedta at § 5-1a får anvendelse mot leder, skal vedkommende avsettes. Ny leder kan konstitueres frem til neste ordinære årsmøte, jf. § 3-4a.

\section{Vedtekter}
\subsection{Endringer av vedtekter}
Endringer av vedtekter og reglementer vedtas på årsmøtet. Endringer vedtas ved kvalifisert flertall (2/3-flertall). Endringsforslag til vedtektene må sendes inn minst åtte dager før årsmøtet.
\end{document}
