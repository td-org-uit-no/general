\documentclass[11pt]{article}

\usepackage[utf8]{inputenc}
\usepackage{enumitem}
\usepackage{fullpage}
\usepackage{appendix}


\setlist[enumerate]{label=\textbf{\alph*})}

\renewcommand{\thesection}{\S\arabic{section}}

\title{\Huge{Vedtekter for\\Tromsøstudentenes Dataforening}}
\date{18.\ september 2023}
\author{}

\begin{document}
\maketitle
\section{Generelle bestemmelser}
\subsection{Navn}
Foreningens navn er Tromsøstudentenes Dataforening, heretter forkortet TD.

\subsection{Formål}
TDs fremste oppgave er å fremme sosialt samhold blant studentene tilhørende institutt for informatikk.
TD skal også tilstrebe å tilby et bredt spekter av både faglige og sosiale arrangementer. 

\subsection{Taushetsplikt}
Medlemmene av TD har taushetsplikt i alle saker og henvendelser der personvern er nødvendig. 

\subsection{Etiske retningslinjer}
Alle medlemmer av TD er underlagt de gjeldende etiske retningslinjer. Alle medlemmer pliktes å sette seg inn i, og handle etter disse. Retningslinjene er definiert i Appendix A av vedtektene.

\section{Årsmøtet}
\subsection{Formål}
Årsmøtet er det øverste organet for medlemmene i TD.

\subsection{Generelle bestemmelser}
\begin{enumerate}
	\item Årsmøtet skal avholdes i vårsemesteret.
	\item Innkalling til årsmøte skal offentliggjøres senest to uker før møtet. Sakspapirer offentliggjøres én uke før møtet. 
	\item Alle studenter som er registrert medlem i TD møter med tale-, forslags- og stemmerett. 
	\item Årsmøtet er vedtaksdyktig dersom det er minimum 10 studenter som beskrevet i §2-2c.
\end{enumerate}

\subsection{Saker som behandles på årsmøtet}
Årsmøtet skal behandle:
\begin{enumerate}
	\item Godkjenning av innkalling
	\item Godkjenning av dagsorden
	\item Valg av ordstyrer, referent og to protokollunderskrivere
	\item Sakene på dagsorden
	\item Valg av nytt styre til TD jf. §4-1 og §4-2. 
\end{enumerate}
Studenter kan sende inn saker til årsmøtet inntil åtte dager før møtet avholdes. Forslag til vedtak som kommer under møtet må komme skriftlig til møteleder.

\subsection{Ekstraordinært årsmøte}
Ekstraordinært årsmøte skal avholdes ved skriftlig krav fra minst 5 medlemmer i TD.\\
Ekstraordinært årsmøte skal avholdes på samme måte som et ordinært årsmøte. Bare saker fremmet før fristen kan behandles. 

\section{Valg og supplering}
\subsection{Valgbare studenter}
Det er kun studenter som er stemmeberettigede ved årsmøtet som kan stille til valg, jf. §2-2c.

\subsection{Gjennomføring av valg}
\begin{enumerate}
	\item Kandidater velges ved simpelt flertall
	\item Dersom minst én i forsamlingen krever det skal det avholdes skriftlig valg
	\item Valg gjennomføres i rekkefølge jamfør opplisting i Kapittel 4
	\item En person kan ikke velges til mer enn ett verv i TDs styre
	\item Alle verv velges for én årsmøteperiode
	\item Kandidater skal gis anledning til en kort presentasjon
\end{enumerate}

\subsection{Årsmøteperiode}
En årsmøteperiode defineres som perioden fra og med 1.\ august inneværende år til og med 31.\ juli påfølgende år.

\subsection{Diversifisering}
Det skal tilstrebes en difersifisering av medlemmene i TD, være seg kjønn og alder/kull.

\subsection{Supplering av TD}
Hvis det oppstår en situasjon der styret ikke er fylt opp på årsmøtet, eller noen fratrer sitt verv i løpet av årsmøteperioden kan det resterende styret selv avgjøre om de ønsker å supplere seg selv eller arrangere et ekstraordinært årsmøte.

\section{Organisasjon}
\subsection{Styret}
Ledelsen i TD heter styret og består av
\begin{enumerate}
	\item én leder
	\item én nestleder
	\item én økonomiansvarlig
	\item én eller to arrangementsansvarlige
	\item én teknisk ansvarlig
	\item én eller to kommunikasjonsansvarlige 
	\item én nettsideansvarlig
    \item én fagansvarlig
\end{enumerate}

Styret og medlemmene vil avgjøre selv hvorvidt det skal være en eller to arrangements-, og kommunikasjonsansvarlige. Dette gjøres ved valg av styret under årsmøtet. Alle styremedlemmer har taushetsplikt om de opplysninger de mottar som styremedlemmer.

\subsection{Arbeidsbeskrivelse}
For alle styremedlemmer forventes det at man møter opp på møter og aktiviteter i regi av TD, samt utfører de arbeidsoppgaver man har påtatt seg. I tillegg gjelder spesifikke arbeidsoppgaver:

\subsubsection{Leder}
Leder fungerer som leder for både styret og TD. Leder fungerer som møteleder på alle møter. Leder har stemmerett i styret og TD. Leder har øverste ansvar for økonomien og aktivitet i regi av TD.

\subsubsection{Nestleder}
Nestleder fungerer som leders stedfortreder. Dersom leder ikke har mulighet til å utføre sine plikter er det nestleders ansvar å utføre disse. Nestleder har stemmerett i styret og TD. Nestleder fungerer vanligvis som referent.

\subsubsection{Økonomisk ansvarlig}
Økonomisk ansvarlig har hovedansvar for TDs økonomi. Dette gjelder både føring av budsjett, regnskap og søknader om økonomisk støtte; gjerne med hjelp fra andre styremedlemmer.

\subsubsection{Arrangement ansvarlig}
Arrangementansvarlig har hovedansvaret for planlegging og gjennomføring av arrangementer i regi av TD. 
Arrangementansvarlig(e) kan også sette ned en arrangementskomité for hvert spesifikke arrangement.

I henhold til §4.1 vil det være mulig å innsette to arrangementansvarlige.

\subsubsection{Teknisk ansvarlig}
Teknisk ansvarlig har hovedansvar for teknisk drift i regi av TD. Dette inkluderer drift av serverrom, andre tekniske installasjoner, samt teknisk bistand ved arrangmenter.

\subsubsection{Kommunikasjons ansvarlig}
Kommunikasjonsansvarlig(e) har hovedansvar for TDs kommunikasjon utad. 
Denne kommunikasjonen inkluderer tilgjengeliggjøring og vedlikehold av informasjonen som finnes om foreningen og dens virke.

I henhold til §4.1 vil det være mulig å innsette to kommunikasjonsansvarlige. Ansvaret vil da deles i to områder; bedriftskommunikasjon og studentkommunikasjon.

\subsubsection{Nettsideansvarlig}
Nettsideansvarlig har hovedansvar for å fremme drift og utvikling av TD sin nettside. Nettsideansvarlig har overordnet ansvar for å legge ut produsert innhold på nettsiden, samt bistå ved komplikasjoner med drift av nettsiden.

\subsubsection{Fagansvarlig}
Fagansvarlig har hovedansvaret for planlegging og gjennomføring av faglige arrangementer i regi av TD. Fagansvarlig kan også sette ned en fagkomité for hvert spesifikke arrangement.

\subsection{Møtevirksomhet}
\subsubsection{Intern møtevirksomhet}
\begin{enumerate}
	\item TD avholder møter etter behov, minst tre ganger pr. semester.
	\item Leder eller minst tre medlemmer kan kreve innkalt til TD-møte.
	\item Møtetidspunkt skal gjøres kjent senest én uke før møtet avholdes.
	\item Forfall skal meldes til leder eller møteinnkaller.
	\item TD er vedtaksdyktige når minst halvparten av styremedlemmene er til stede.
	\item Nestleder eller annen referent har ansvar for å føre referat og at det sendes ut til medlemmene senest to dager etter møtet. Det er en høringsfrist på to virkedager. Etter fristen skal referatet offentliggjøres. Dermed kan referatet ikke inneholde informasjon unntatt fra offentligheten.
	\item Møtene er åpne, dersom man ikke vedtar å lukke ett møte.
	\item Simpelt flertall er tilstrekkelig for å gjøre vedtak i en sak. Ved stemmelikhet har leder dobbeltstemme.
	\item Avstemning skjer ved håndsopprekning, dersom ingen krever skriftlig avstemning.
	\item Dersom et medlem er sterkt uenig i et vedtak, kan vedkommende kreve å få det protokollført.
	\item Styret kan gis fullmakt til å gjøre vedtak i hastesaker.
\end{enumerate}

\subsubsection{Årsmøtet}
TD skal avholde et årsmøte én gang tidlig på høstsemesteret. Hensikten med årsmøtet er å velge et nytt styre, samt å opprettholde kontakt med medlemsmassen. Etter endt årsmøte skal det sendes inn styreendringer til Brønnøysundregistrene i henhold til deres instrukser.

\subsection{Bedriftaktiviteter}
TD skal ta et honorar for alle typer aktiviteter med bedrifter der TD benytter sin posisjon for å sette bedriften i kontakt med TDs studenter. Ved spesielle tilfeller der styret ved kvalifisert flertall (2/3-flertall) er enige om et unntak kan dette gjøres, noe som må reflekteres i møtereferatet. TD-styret fastsetter honorarsatser. 

\subsection{Æresmedlem}
Æresmedlem er den høyeste utmerkelsen i foreningen. Et æresmedlem gis utover et vanlig medlemskap av foreningen heder og ære for sitt virke for foreningen.

\subsubsection{Utnevnelse av æresmedlem}
Æresmedlem skal nomineres i forkant av et årsmøte, hvor en vurdering vil bli gjort av styremedlemmene i TD. Æresmedlem er ikke begrenset til medlemmer eller tidligere medlemmer av foreningen, men utnevnes etter vurdering av følgende kriterier:

\begin{enumerate}
	\item En sterk ressurs som har utøvd betydelig arbeid utover forventningene av sin rolle i foreningen.
	\item Har gjennom sitt arbeid lagt til rette særlig fortjeneste for foreningen.
\end{enumerate}

Det presiseres viktighet i vurderingen av kriteriene, slik at inflasjon av æresmedlemskap unngås.

\subsection{Medlemskap}
Medlemskap av Tromsøstudentenes Dataforening er forbeholdt studenter ved studieretninger tilknyttet Institutt for Informatikk, UiT. Dette gjelder også tidligere studenter ved ovennevnte studieretninger. 

\subsubsection{Kontingent}
Det betales ikke medlemskontingent for medlemmer av Tromsøstudentenes Dataforening.

\subsubsection{Oppdatering av medlemsstatus}
Tromsøstudentenes Dataforening må årlig oppdatere sin medlemsinformasjon.
Det forventes at medlemmer i Tromsøstudentenes Dataforening oppdaterer sin status som enten uteksaminert, kommende eller aktiv student, samt årskull og annen informasjon som angår Tromsøstudentenes Dataforening.

\section{Sanksjoner}
\subsection{Iverksettelse av sanksjoner}
\begin{enumerate}
	\item Sanksjoner skal iverksettes mot de som fremstår som vesentlig uegnet til vervet de innehar.
	\item Før sanksjoner blir vedtatt skal den det gjelder få en advarsel om gjeldende forhold. Dette kan fravikes i særs grove tilfeller.
	\item Før sanksjoner blir iverksatt skal den det gjelder få anledning til å uttale seg for TD.
	\item Den det sanksjoneres mot har krav på skriftlig begrunnelse for vedtaket.
\end{enumerate}

\subsection{Mot medlemmer av TD}
\begin{enumerate}
	\item Dersom det er simpelt flertall i styret for å vedta at § 5-1a får anvendelse mot et styremedlem, skal vedkommende anmodes om å fratre. Leder skal foreta denne anmodningen, selv om vedkommende stemte mot vedtaket.
	\item Dersom styremedlemmet velger å fratre kan styret supplere seg selv ved å konstituere et nytt medlem frem til neste ordinære årsmøte, jf. § 3-4.
	\item Dersom medlemmet ikke ønsker å fratre skal det innkalles til ekstraordinært årsmøte, og reises mistillitsforslag der.
\end{enumerate}

\subsection{Mot leder av TD}
Dersom det foreligger 2/3-flertall i styret for å vedta at § 5-1a får anvendelse mot leder, skal vedkommende avsettes. Ny leder kan konstitueres frem til neste ordinære årsmøte, jf. § 3-4a.

\section{Vedtekter}
\subsection{Endringer av vedtekter}
Endringer av vedtekter og reglementer vedtas på årsmøtet. Endringer vedtas ved kvalifisert flertall (2/3-flertall). Endringsforslag til vedtektene må sendes inn minst åtte dager før årsmøtet.
\newpage

\renewcommand{\thepage}{}


\appendix
\section{TDs Etiske Retningslinjer}

\subsubsection{Innledning}
Ved å følge disse etiske retningslinjene bidrar du til å styrke TD som en trygg, inkluderende og respektfull studentforening ved Universitetet i Tromsø. Din innsats er avgjørende for å opprettholde vår gode kultur og omdømme. Sammen kan vi skape et positivt og berikende studiemiljø for alle våre medlemmer.
 
\subsubsection{Formål}
Formålet med våre etiske retningslinjer er å definere hvordan vi, som medlemmer av TD, skal opptre mot hverandre og i samfunnet. Dette er med på å skape et trygt og inkluderende miljø for alle våre medlemmer, samtidig som vi sikrer integriteten og omdømmet til foreningen.
 
\subsubsection{Hvem gjelder retningslinjene for?}
TD sine etiske retningslinjer gjelder for alle medlemmer av foreningen, inkludert nåværende og tidligere studenter som har vært eller er medlemmer. Det er et felles ansvar for alle medlemmer å følge og håndheve disse retningslinjene.
 
\subsubsection{Varsling}
Alle medlemmer har en plikt til å varsle om brudd på disse retningslinjene. Dette er viktig for å sikre et trygt og inkluderende miljø og for å opprettholde god kultur i TD. Hvordan varsling skal skje, er beskrevet senere i dokumentet.
 
\subsection{Generelle Prinsipper}
 \begin{itemize}
     \item[A.1.1] \textbf{Overholdelse av Lover og Regler}
     \newline
     Som medlem av TD forventes det at du følger norsk lov og regelverk, både innenfor og utenfor universitetet. Ditt medlemskap i foreningen gir deg muligheten til å påvirke samfunnet rundt deg og styrke samholdet med dine medstudenter.

     \item[A.1.2] \textbf{Respekt og Inkludering} 
     \newline
     Medlemmer av TD skal vise respekt for hverandre og behandle alle medmennesker med verdighet og inkludering. Vi forventer at våre medlemmer ikke diskriminerer basert på kjønn, religion, nasjonalitet, alder, funksjonsevne, seksuell orientering eller andre identitetsfaktorer.

     \item[A.1.3] \textbf{Eiendomsbehandling}
     \newline
     Medlemmer av TD skal behandle foreningens eiendom med respekt og sørge for at den brukes på en ansvarlig måte.     
 \end{itemize}

\subsection{Oppførsel og Kultur i TD}
TD Forening forplikter seg til å fremme en positiv og inkluderende kultur blant sine medlemmer. Vi legger vekt på å skape et trygt og respektfullt studiemiljø der alle føler seg velkomne og ivaretatt. I TD skal vi:
 \begin{itemize}
     \item[A.2.1] \textbf{Respekt og Omtanke}
     \newline
     Vi skal ha en kultur der alle medlemmer behandler hverandre med respekt og toleranse. Vi oppfordrer til å hjelpe hverandre og vise omtanke i alle våre aktiviteter. Diskriminering, nedsettende kommentarer og trakassering av noen form er uakseptabelt og vil ikke bli tolerert i TD.

     \item[A.2.2] \textbf{Nulltoleranse mot Mobbing og Trakassering}
     \newline
     Mobbing og trakassering er absolutt uakseptabelt i TD. Vi tar alle tilfeller av mobbing, diskriminering eller trakassering på alvor. Medlemmer som opplever slik atferd skal føle seg trygge på å melde fra, og vi vil håndtere slike saker med høy prioritet og konfidensialitet.

     \item[A.2.3] \textbf{Opptreden på TD-arrangementer}
     \newline
     TD arrangerer en rekke sosiale og faglige arrangementer både internt og i samarbeid med eksterne partnere for våre medlemmer. Denne seksjonen beskriver hvordan våre etiske retningslinjer gjelder for slike arrangementer. I tillegg forventes det at våre medlemmer følger arrangementets regler og retningslinjer som er fastsatt av TD.
     \begin{enumerate}
         \item[A.2.3.1] \textbf{Generelt}
         \newline
         Ved deltakelse på alle TD-arrangementer forventer vi at alle medlemmer opptrer med god oppførsel og vanlig folkeskikk. Deltakerne skal vise respekt for hverandre og arrangørene av arrangementet, og følge de angitte arrangementreglene. Vi aksepterer ikke hærverk, og tilfeller av hærverk vil bli møtt med sanksjoner.

         \item[A.2.3.2] \textbf{Med andre studentorganisasjoner}
         \newline
         Når våre medlemmer deltar på arrangementer som er organisert av eller i samarbeid med andre studentorganisasjoner, forventer vi at de opprettholder de samme høye standardene for oppførsel som beskrevet i seksjon Generelt over. Respekt skal vises til alle deltakere, uavhengig av hvilken organisasjon de tilhører. Personer utenfor TD kan rapportere saker angående våre medlemmer.\\\\
         Hvis et medlem som representerer TD er involvert i eller utfører handlinger som er i strid med våre etiske retningslinjer under arrangementer organisert av andre studentorganisasjoner, vil TD behandle saken og pålegge sanksjoner hvis nødvendig.\\\\
         Hvis et TD-medlem melder fra om en hendelse som involverer noen fra en annen studentorganisasjon, har TD ikke myndighet til å pålegge sanksjoner. Vi vil imidlertid veilede medlemmet best mulig for å henvise saken til riktig instans. Dette kan for eksempel være linjeforeningen til den andre studenten eller UiT, og det oppfordres til å rapportere om upassende hendelser.

         \item[A.2.3.3] \textbf{Med bedrifter} 
         \newline
         Ved bedriftsarrangementer forventer vi at våre medlemmer viser respekt for at bedriftene investerer tid og ressurser i å presentere seg som mulige arbeidsgivere for TD-medlemmer. Deltakerne skal oppføre seg som gode representanter for TD.\\\\
         Hvis det kommer varsler om kritikkverdige forhold angående en representant fra en bedrift, har TD ikke myndighet til å pålegge sanksjoner. Saken vil imidlertid bli videreført til rett instans i bedriften det gjelder.
     \end{enumerate}
 \end{itemize}
 
\subsection{Konsekvenser for Uakseptabel Oppførsel}
 
\subsubsection{Ved Arrangementer}
Ved arrangementer skilles det mellom to typer reaksjoner: umiddelbare sanksjoner og sanksjoner i etterkant av arrangementer.
 \begin{itemize}
     \item[A.3.1.1] \textbf{Umiddelbare Sanksjoner}
     \newline
     Oppførsel som oppleves som uakseptabel av arrangørene kan føre til umiddelbare konsekvenser, inkludert:
     \begin{enumerate}
         \item Muntlige Advarsler: Medlemmet kan bli gitt en muntlig advarsel under eller etter arrangementet.
         \item Midlertidig Bortvisning: Medlemmet kan midlertidig utestenges fra arrangementet.
         \item Utdeling av Prikk: TDs nettsted kan registrere prikker mot medlemmets oppførsel.
     \end{enumerate}
     \item[A.3.1.2] \textbf{Sanksjoner i Etterkant}
     \newline
     Hvis det kommer varsler om kritikkverdige forhold i forbindelse med et arrangement, kan sanksjoner med høyere alvorlighetsgrad vurderes i etterkant.

 \end{itemize}
 
\subsection{Konsekvenser for Brudd på Retningslinjene}
TD tar brudd på sine etiske retningslinjer alvorlig. Konsekvensene for brudd kan variere avhengig av alvorlighetsgraden av bruddet. Mulige sanksjoner kan inkludere:
\begin{enumerate}
    \item Advarsler: Medlemmet kan motta en skriftlig advarsel.
    \item Suspensjon: Medlemmet kan midlertidig utestenges fra foreningen.
    \item Eksklusjon: Medlemmet kan ekskluderes fra foreningen.
    \item Anmeldelse til Myndighetene: Alvorlige brudd kan anmeldes til relevante myndigheter i samsvar med gjeldende lover og regler.
\end{enumerate}
 
\subsection{Ulovlige handlinger}
Dersom det oppstår og meldes om ulovlige handlinger som er knyttet til medlemmer av TD, skal disse i hovedsak håndteres av rettshåndhevelsesmyndighetene, og saken skal overføres til politiet. Samtidig vil TD vurdere å pålegge egne sanksjoner basert på alvorlighetsgraden av overtredelsen.
 
\subsection{Representasjon for TD}
For medlemmer som utfører arbeid på vegne av TD og dermed kan sies å være en representant for foreningen, gjelder følgende retningslinjer i tillegg til våre øvrige bestemmelser:
 \begin{itemize}
     \item[A.6.1] \textbf{Arrangementer}
     \newline
     Medlemmer som arrangerer ting på vegne av TD og som kan anses som i tjeneste for foreningen, forventes å utføre sine oppgaver på en ansvarlig og profesjonell måte. Vi legger stor vekt på tilliten som er plassert i arrangører for å tilby sosiale og faglige aktiviteter til det beste for TD og å forvalte foreningens midler på en forsvarlig måte. Derfor må alle beslutninger tas med TDs beste interesser i tankene. Ansvarlige arrangører må utvise ansvarlig oppførsel og unngå situasjoner der de ikke kan gjennomføre arrangementet på en tilfredsstillende måte.

     \item[A.6.2] \textbf{Bruk av TDs eiendom og utstyr} 
     \newline
     Utstyr som eies av TD og som brukes i forbindelse med våre arrangementer, skal alltid holdes i god stand. Alt som tilhører TD skal behandles og brukes i samsvar med gjeldende regler og retningslinjer. Dersom noen ønsker å bruke TDs utstyr til privat bruk, må dette avklares med ansvarlig for utstyret og skje i samsvar med fastsatte retningslinjer. Det er ikke tillatt å utnytte TDs utstyr på en måte som hindrer andre TD-medlemmers muligheter til å bruke utstyret.\\\\
     Ved ødeleggelse eller skade på TDs utstyr kan de ansvarlige pålegges erstatningskrav, i tillegg til andre sanksjoner hvis det anses som nødvendig. Vi skiller her mellom to ulike situasjoner:
     \begin{enumerate}
         \item I tjeneste for TD: Hvis skaden oppsto mens medlemmet utførte oppgaver for TD, dekkes skadene av foreningen. Dette gjelder så lenge det ikke er fremvist uaktsomhet.
         \item Privat bruk: Hvis utstyret ble skadet utenfor arbeid på vegne av TD, vil medlemmet selv være ansvarlig for økonomisk erstatning.
     \end{enumerate}

     \item[A.6.3] \textbf{Taushetsplikt og Konfidensialitet} 
     \newline
     Taushetsplikten er en grunnleggende forutsetning for å bevare åpenheten og tilliten i TD. Personer som innehar verv i TD er underlagt taushetsplikt når det gjelder konfidensiell informasjon som er tilegnet gjennom vervet.\\\\
     Informasjon som medlemmer får tilgang til gjennom sitt medlemskap eller verv i TD, som kan skade foreningens konkurransefortrinn, skal ikke deles. Dette inkluderer informasjon om priser, prosedyrer, avtaler og kontrakter med eksterne parter. Dette gjelder også informasjon som er tilgjengelig for medlemmer, men som ikke skal deles offentlig.

     \item[A.6.4] \textbf{Lederverv i TD}
     \newline
     For medlemmer som innehar lederverv i TD, for eksempel medlemmer av styret, komitéledere og andre lederposisjoner, er det avgjørende å være gode forbilder og representanter for foreningen. Avgjørelser som blir tatt må alltid ha TD beste interesser som grunnlag. Det er viktig å være klar over den posisjonen man innehar og beslutningsmyndigheten som er tildelt, og ikke utnytte dette til personlig vinning.
 \end{itemize} 
 
\subsection{Varsling}
Varsling er en viktig mekanisme for å rapportere brudd på disse etiske retningslinjene. TD oppfordrer alle medlemmer til å rapportere ethvert brudd de er vitne til eller blir utsatt for. Varslere vil bli beskyttet mot gjengjeldelse i samsvar med gjeldende lover og regler. Et varsel burde inneholde svar på følgende spørsmål:
 \begin{enumerate}
     \item Hva har skjedd?
     \item Hvem er involvert?
     \item Hvor skjedde det?
     \item Når skjedde det?
     \item Har det skjedd flere ganger?
     \item Var det vitner tilstede?
 \end{enumerate}
 
Varsling kan gjøres skriftlig til TDs styre eller til en uavhengig instans som er utpekt for å håndtere slike saker. Når en varsling mottas, vil den bli behandlet konfidensielt, og det vil bli gjennomført en grundig undersøkelse av påstandene. Påstander som viser seg å være sanne, vil føre til passende tiltak i henhold til TDs retningslinjer og relevante lover.
 
\subsection{Oppdatering av Retningslinjene} 
De etiske retningslinjene skal gjennomgås og oppdateres etter behov for å sikre at de fortsatt reflekterer våre verdier og foreningens formål. Medlemmer vil bli varslet om eventuelle endringer i retningslinjene og oppfordres til å gjennomgå dem regelmessig.\\\\
 
**Sist oppdatert:** \date{18.\ september 2023}



\end{document}
